\documentclass[a4paper,sans]{moderncv}

\usepackage[english,russian]{babel}

\ifXeTeX
  \usepackage{polyglossia}
  %% устанавливает главный язык документа
  \setdefaultlanguage{english}
  %% устанавливает второй язык документа
  \setotherlanguage{russian}
  \setmainfont{Noto Sans}
  \setsansfont{Noto Sans}
  \setromanfont{Crimson}
\fi

\usepackage{graphicx}
\graphicspath{ {img} }

\usepackage{geometry}
\geometry{left=15mm, top=15mm, right=20mm, bottom=20mm}

\moderncvtheme{classic}
\moderncvcolor{black}

%%%%%%%%%%%%%%%%
% Contact info %
%%%%%%%%%%%%%%%%
\name{Artsiom}{Karakin}
\title{PCB Layout Engineer \&\\ Embedded Programmer}
%\phone[mobile]{+375292560234}
\email{karakin2000@gmail.com}
\social[github]{artsi0m}
%\social[matrix]{artsi0m:matrix.org}
\social[matrix]{artsi0m:nope.chat}
\social[telegram]{artsi0m}
\photo[110pt][1pt]{Bsuir-photo.jpg}

\begin{document}

\maketitle

\section{Work Experience}
\cventry{2024-06-07 2024-07-10}{PCB Layout Engineer}
% \cventry{2024}{PCB Layout Engineer}
%{ООО «РТЕ Сервис» rte.by}{Minsk}{}{
{RTE Service LLC rte.by}{Minsk}{}{ I have completed an apprenticeship
  and created a project of RS232 to RS485 converter. I learned how to
  design working PCB prototypes and improve them with corrections from
  the production team.}


%%% Local Variables:
%%% mode: LaTeX
%%% TeX-master: "Karakin-PCB-layout-and-embedded-en"
%%% End:


\section{Education}
\cventry{2025}{Radio-electronics Engineer}
{Belarusian State University of Informatics and Radio-electronics bsuir.by}
{Minsk}{Bachelor}{ In my final year. I'll graduate in 2025.\\
  I got a 10/10 in the microcontroller course.
  I also prepared reports on CAD programs for the
  for the university science conference.
  I have learnt how to use EDA CAD programs such as KiCAD or CAE programs such as COMSOL
  Multiphysics.\\
  The subject of my bachelor thesis will be the servo drive test system.}

\section{Key skills \& Completed Projects}
\subsection{PCB design with KiCAD}

\cvitem{RS232 to RS485 Converter}{
  Two layered PCB design with structured element list
  done on apprenticeship.
  \url{https://github.com/artsi0m/6sem_practise}}

\cvitem{Hacker-space Tube PCB}{
  Single layer PCB design
  with fabrication files saved in git repository.
  Self-education project done in Minsk Hacker-space.
  \url{https://github.com/artsi0m/HS_tube_KiCAD}
}

\subsection{AVR assembly programming}

\cvitem{Segment Display with Pinboard 2}{
  BCD to 7 segment decoder written in Atmel AVR-assembly
  for Pinboard 2 devboard.
  Learned how to write in assembly with Atmel studio and flash
  program to micro-controller using avrdude.
  \url{https://github.com/artsi0m/lab1_segment_display}
}

\subsection{C Programming}

\cvitem{argcalc}{
  Re-implementation of an expr(1) utility from UNIX.
  Simple command line calculator using Dijkstra's Shunting Yard algorithm.
  \url{https://github.com/artsi0m/argcalc}
}

% current date
\vfill
Last edited: \today

\end{document}

%%% Local Variables:
%%% mode: LaTeX
%%% TeX-master: t
%%% TeX-engine: xetex
%%% End:
