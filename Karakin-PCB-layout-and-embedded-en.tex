%%% Local Variables:
%%% mode: LaTeX
%%% TeX-master: t
%%% TeX-engine: xetex
%%% End:
\documentclass[a4paper,sans]{moderncv}

\usepackage[english,russian]{babel}

\ifXeTeX
  \usepackage{polyglossia}
  %% устанавливает главный язык документа
  \setdefaultlanguage{english}
  %% устанавливает второй язык документа
  \setotherlanguage{russian}
  \setmainfont{Noto Sans}
  \setsansfont{Noto Sans}
  \setromanfont{Crimson}
\fi

\usepackage{graphicx}
\graphicspath{ {img} }

\usepackage{geometry}
\geometry{left=15mm, top=15mm, right=20mm, bottom=20mm}

\moderncvtheme{classic}
\moderncvcolor{black}

%%%%%%%%%%%%%%%%
% Contact info %
%%%%%%%%%%%%%%%%
\name{Artsiom}{Karakin}
\title{PCB Layout Engineer \&\\ Embedded Programmer}
%\phone[mobile]{+375292560234}
\email{karakin2000@gmail.com}
\social[github]{artsi0m}
%\social[matrix]{artsi0m:matrix.org}
\social[matrix]{artsi0m:nope.chat}
\social[telegram]{artsi0m}
\photo[110pt][1pt]{Bsuir-photo.jpg}

\begin{document}

\maketitle



\section{Work Experience}
\cventry{2024-06-07--2024-07-10}{PCB Layout Engineer}{\textrm{ООО РТЕ Сервис} rte.by}
    {Minsk}{}{Completed an apprenticeship}

\section{Education}
\cventry{2025}{Radioelectronics Engineer}
{Belarussian State University of Informatics and Radioelectronics bsuir.by}{Minsk}
{Bachelor}{In my senior year. I'll graduate by 2025.}
   
\section{Key skills \& Completed Projects}
\subsection{PCB design with KiCAD}

\cvitem{RS232 to RS485 Converter}{
  Two layred PCB design with structured element list
  done on apprenteceship. Learned that I should always
  verify my project in simulator.
  \url{https://github.com/artsi0m/6sem_practise}}
   
\cvitem{Hackerspace Tube PCB}{
  \textbf{Single} layer PCB design
  with fabrication files saved in git repo.
  Self-education project done in Minsk Hackerspace.
  \url{https://github.com/artsi0m/HS_tube_KiCAD}
}

\subsection{AVR assembly programming}

\cvitem{Segment Display with Pinboard 2}{
  BCD to 7 segment decoder written in Atmel AVR-assembly
  for Pinboard 2 devboard.
  Learned how to write in assembly with Atmel studio and flash
  programm to microcontroller using avrdude.
  \url{https://github.com/artsi0m/lab1_segment_display}
}

\subsection{C Programming}

\cvitem{argcalc}{
  Reimplementation of an expr(1) utility from UNIX .
  Simple command line calculator using Dejkstra's Shunting Yard algorithm.
  \url{https://github.com/artsi0m/argcalc}
}

\end{document}
